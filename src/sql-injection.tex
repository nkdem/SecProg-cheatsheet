\wde{Injection}{
    Succinctly defined as part of CWE-74: Improper Neutralization of Special Elements in Output Used by a Downstream Component, i.e. 
    \begin{quote}
    ALWAYS CHECK YOUR INPUTS
    \end{quote}
}

\wde{Metadata}{
    Metadata accompanies the main data and represents additional information about it, such as how to display textual strings or where a string ends. In the context of SQL injection, metadata can influence how queries are constructed and processed.
}

\wde{In-band Representation}{
    In-band representation embeds metadata into the data stream itself, such as using special characters within SQL queries. This can lead to injection vulnerabilities if user input is not properly sanitized.
}

\wde{Out-of-band Representation}{
    Out-of-band representation separates metadata from data, making it less susceptible to injection attacks. For example, using prepared statements allows the SQL engine to distinguish between data and commands.
}

\wde{Input Validation}{
    Input validation can be achieved through blacklisting and whitelisting:
    \begin{itemize}
        \item \textbf{Blacklisting}: Keeping a list of forbidden characters or patterns and rejecting inputs that contain them.
        \item \textbf{Whitelisting}: Keeping a list of allowed characters or patterns and rejecting inputs that do not match.
    \end{itemize}
}

\wde{SQL Injection}{
    SQL Injection (CWE-89) is a command injection (CWE-77) that allows attackers to execute arbitrary SQL queries through user input. This can lead to unauthorized access to the database, data leakage, and data corruption.
}

\wde{Common SQL Injection Techniques}{
    SQL injection techniques can be classified into several categories:
    \begin{itemize}
        \item **Tautologies**: Injecting code that always evaluates to true.
        \item **Illegal/Incorrect Queries**: Causing run-time errors to learn information from error messages.
        \item **Union Queries**: Combining results from multiple queries to extract additional data.
        \item **Piggy-Backed Queries**: Executing multiple queries in a single request.
        \item **Inference Pairs**: Using differences in responses to infer information.
        \item **Stored Procedures**: Exploiting vulnerabilities in stored procedures to execute arbitrary SQL commands.
    \end{itemize}
}

\wde{Prevention and Detection}{
    To prevent SQL injection vulnerabilities, developers can:
    \begin{itemize}
        \item Use prepared statements and parameterized queries.
        \item Implement rigorous input validation and sanitization.
        \item Employ web application firewalls (WAFs) to filter malicious requests.
        \item Regularly test and audit code for vulnerabilities.
    \end{itemize}
}
