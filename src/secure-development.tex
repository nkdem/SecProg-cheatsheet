\wde{CVE Format and Meaning}{
    Common Vulnerabilities and Exposures (CVE) is a list of publicly disclosed cybersecurity vulnerabilities and exposures. Each CVE entry includes:
    \begin{itemize}
        \item CVE ID: a unique identifier for the vulnerability.
        \item Description: a brief summary of the vulnerability.
        \item References: links to advisories or reports that provide more details.
    \end{itemize}
}

\wde{CVSS - Common Vulnerability Scoring System}{
    The Common Vulnerability Scoring System (CVSS) is a standard for assessing the severity of software vulnerabilities. It provides a numerical score based on the following metrics:
    \begin{itemize}
        \item \textbf{Base Metrics}: Characteristics of the vulnerability itself.
        \item \textbf{Temporal Metrics}: Characteristics that change over time (e.g., exploit availability).
        \item \textbf{Environmental Metrics}: Characteristics specific to the organization's environment.
    \end{itemize}
    The CVSS score helps organizations prioritize and respond to vulnerabilities based on their severity.
}
\wde{CVSS Base Metrics}{
    The CVSS Base Metrics include:
    \begin{itemize}
        \item \textbf{Attack Vector}: How the vulnerability is exploited \textit{(Local, Adjacent, Network)}.
        \item \textbf{Attack Complexity}: The level of expertise required to exploit the vulnerability \textit{(Low, High)}.
        \item \textbf{Privileges Required}: The level of privileges an attacker needs to exploit the vulnerability \textit{(None, Low, High)}.
        \item \textbf{User Interaction}: Whether user interaction is required to exploit the vulnerability \textit{(None, Required)}.
        \item \textbf{Scope}: Whether the vulnerability impacts the vulnerable component only or can affect other components \textit{(Unchanged, Changed)}.
        \item \textbf{Confidentiality, Integrity, Availability Impact}: The impact on the security properties of confidentiality, integrity, and availability \textit{(None, Low, High)}.
    \end{itemize}
}
\wde{Common Weakness Enumeration (CWE)}{
    The CWE is a community-developed list of common software security weaknesses. It provides a classification of software vulnerabilities to help developers understand and mitigate risks.
}
\wde{Security Properties}{
    Key security properties that must be considered in secure programming include:
    \begin{enumerate}
        \item \textbf{Confidentiality}: Data is only available to the people intended to access it.
        \item \textbf{Integrity}: Data and system resources are only changed in appropriate ways by appropriate people.
        \item \textbf{Availability}: Systems are ready when needed and perform acceptably.
        \item \textbf{Authentication}: The identity of users is established (or you're willing to accept anonymous users).
        \item \textbf{Authorization}: Users are explicitly allowed or denied access to resources.
        \item \textbf{Nonrepudiation}: Users can't perform an action and later deny performing it.
    \end{enumerate}
}

\wde{STRIDE}{
    STRIDE is a mnemonic for categories of security threats, along with the corresponding security properties that each threat type attacks:
    \begin{itemize}
        \item \textbf{Spoofing}: Attacker pretends to be someone else. \textit{(Attacks Authentication)}
        \item \textbf{Tampering}: Attacker alters data or settings. \textit{(Attacks Integrity)}
        \item \textbf{Repudiation}: User can deny making an attack. \textit{(Attacks Non-repudiation)}
        \item \textbf{Information Disclosure}: Loss of personal information. \textit{(Attacks Confidentiality)}
        \item \textbf{Denial of Service}: Preventing proper site operation. \textit{(Attacks Availability)}
        \item \textbf{Elevation of Privilege}: User gains higher privileges. \textit{(Attacks Authorization)}
    \end{itemize}
}


\wde{Saltzer's Classic Principles}{
    Saltzer and Schroeder's principles provide guidelines for secure design:
    \begin{enumerate}
        \item \textbf{Economy of Mechanism}: Keep the design simple.
        \item \textbf{Fail-Safe Defaults}: Default to secure configurations; fail closed with no single point of failure.
        \item \textbf{Complete Mediation}: Check permissions on every access; ensure that all access requests are verified.
        \item \textbf{Open Design}: Assume that attackers have access to the source code and specifications; design should not rely on obscurity.
        \item \textbf{Separation of Privilege}: Require multiple conditions to grant access to sensitive operations; don’t permit an operation based on a single condition.
        \item \textbf{Least Privilege}: Grant only the minimum privileges necessary for users or processes; no more privileges than what is needed.
        \item \textbf{Least Common Mechanism}: Minimize shared resources; be cautious of mechanisms that are shared among users.
        \item \textbf{Psychological Acceptability}: Ensure that security measures are user-friendly and do not hinder usability; security should be intuitive for users.
    \end{enumerate}
}


\wde{McGraw's Three Pillars}{
    McGraw's approach is built on three pillars:
    \begin{enumerate}
        \item \textbf{Applied Risk Management}: Identify, rank, and track risks using threat modeling to uncover security risks.
        \item \textbf{Software Security Touchpoints}: Integrate security-related activities throughout the software development lifecycle.
        \item \textbf{Knowledge}: Leverage existing knowledge, programming guidelines, and known exploits to enhance security practices.
    \end{enumerate}
}

\wde{McGraw's Seven Touchpoints}{
    McGraw identified 7 touchpoints that could be integrated in the traditional software development lifecycle (SDLC)
    \begin{enumerate}
        \item \textbf{Abuse Cases}: Identify potential misuse scenarios.
            \begin{itemize}
                \item Artifacts: Use case documents, threat models.
                \item Problems: Failure to consider all misuse scenarios can lead to unaddressed vulnerabilities.
                \item Bad Example: Ignoring input validation in a web application leading to SQL injection.
            \end{itemize}
        \item \textbf{Security Requirements}: Define security needs early in the process.
            \begin{itemize}
                \item Artifacts: Security requirement specifications, risk assessment reports.
                \item Problems: Vague or incomplete security requirements can result in inadequate protection.
                \item Bad Example: A requirement stating "the application should be secure" without specifics on encryption or access controls.
            \end{itemize}
        \item \textbf{Architectural Risk Analysis}: Assess risks in design and architecture.
            \begin{itemize}
                \item Artifacts: Architecture diagrams, risk analysis documents, design specifications.
                \item Problems: Failing to identify security flaws in the architecture can lead to critical vulnerabilities.
                \item Bad Example: Designing a system without considering the security of third-party components.
            \end{itemize}
        \item \textbf{Risk-Based Security Testing}: Focus on testing based on risk analysis.
            \begin{itemize}
                \item Artifacts: Test plans, test cases, risk assessment matrices.
                \item Problems: Inadequate testing of high-risk areas may leave critical vulnerabilities untested.
                \item Bad Example: Conducting extensive tests on low-risk features while neglecting authentication mechanisms.
            \end{itemize}
        \item \textbf{Code Review}: Inspect code for security vulnerabilities.
            \begin{itemize}
                \item Artifacts: Source code, code review checklists, static analysis reports.
                \item Problems: Insufficient or poorly structured code reviews can miss significant vulnerabilities.
                \item Bad Example: Relying solely on automated tools without manual review, leading to overlooked security issues.
            \end{itemize}
        \item \textbf{Penetration Testing}: Test the system for vulnerabilities in a real-world context.
            \begin{itemize}
                \item Artifacts: Penetration testing reports, vulnerability assessment tools, exploitation scripts.
                \item Problems: Conducting penetration tests without understanding the system can result in incomplete assessments.
                \item Bad Example: External testers not familiar with the application’s architecture, leading to ineffective testing.
            \end{itemize}
        \item \textbf{Security Operations}: Manage security during the deployment phase.
            \begin{itemize}
                \item Artifacts: Security operation procedures, incident response plans, monitoring logs.
                \item Problems: Lack of continuous monitoring can lead to undetected security incidents.
                \item Bad Example: Deploying an application without proper logging and monitoring, making incident response difficult.
            \end{itemize}
    \end{enumerate}
}

\wt{Dijkstra's Observation on Testing}{
    Dijkstra's famous remark states:
    \begin{quote}
        "Testing shows the presence, not the absence of bugs."
    \end{quote}
    This highlights the importance of comprehensive testing strategies and the understanding that passing tests do not guarantee that no bugs exist in the software.
}

\wde{Building Security In Maturity Model (BSIMM)}{
It is a blueprint/software security framework for following good practices of software development. It is data-driven and so the practices are collated by examining the strategies that companies are utilising to produce secure software.
% BSIMM consists of a total of 126 activities that are assigned maturity levels (1-3), where level 1 indicates the most mature practices that are commonly adopted by companies. 
The framework is composed of four domains: 
Governance, Intelligence, SSDL Touchpoints, and Deployment, with each domain containing three practices (for a total of 12 practices).
\begin{itemize}
    \item Governance: Establishing security policies and practices.
    \begin{itemize}
        \item Strategy and metrics
        \item Compliance and policy.
        \item Training.
    \end{itemize}
    \item Intelligence: Gathering security metrics and threat intelligence.
    \begin{itemize}
        \item Attack models.
        \item Security features.
        \item Standards and requirements.
    \end{itemize}
    \item Development: Integrating security into the development process.
    \begin{itemize}
        \item Code review.
        \item Security testing.
        \item Architecture analysis.
    \end{itemize}
    \item Deployment: Ensuring secure deployment and operations.
    \begin{itemize}
        \item Environment hardening.
        \item Software environment.
        \item Incident management.
    \end{itemize}
\end{itemize}
}

